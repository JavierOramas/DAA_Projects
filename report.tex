\documentclass{article}
\usepackage{listings}
% Packages
\usepackage[margin=1in]{geometry} % set margin size
\usepackage{color}

\definecolor{dkgreen}{rgb}{0,0.6,0}
\definecolor{gray}{rgb}{0.5,0.5,0.5}
\definecolor{mauve}{rgb}{0.58,0,0.82}

\lstset{frame=tb,
  language=Python,
  aboveskip=3mm,
  belowskip=3mm,
  showstringspaces=false,
  columns=flexible,
  basicstyle={\small\ttfamily},
  numbers=none,
  numberstyle=\tiny\color{gray},
  keywordstyle=\color{blue},
  commentstyle=\color{dkgreen},
  stringstyle=\color{mauve},
  breaklines=true,
  breakatwhitespace=true,
  tabsize=3
}
% Title and author information
\title{Primer Proyecto de DAA: Procrastinación++}
\author{
    Javier A. Oramas López \\
    Eduardo García Maleta
}
\date{\today}

\begin{document}

\maketitle

\section{Descripción del problema}

El problema pide, dada una cadena de Caracteres $S$, y una cadena $T$
Contar todas las cadenas $A$ que tienen a $T$ como prefijo y que se forman a partir de el siguiente procedimiento:

Comenzamos con $A = ''$; 
Se toma el primer caracter de S, este caractrer se puede colocar al inicio o al final de la cadena A.
Se repite este proceso hasta que no queden Caracteres en $S$.

\subsection{Enfoque Backtracking}
La solución más ingenua que se puede generar consiste en simular el proceso descrito anteriormente,
Se toma cada caracter de S y se inserta primero delante de la cadena A, se llama recursivamente al método con la cadena nueva.
Una vez termina el proceso, se verifica si la cadena que se formó contiene a T como prefijo, si es así se cuenta una solución.
Una vez se retorna, se elimina el caracter insertado al inicio de la cadena y se coloca al final, análogamente se recorre todo el camino hasta el final.

La correctitud de este algoritmo se infiere de la definición del problema, dado que está simulando lo descrito el enunciado.

La complejidad de estee algorirtmo es $O(2^{|S|})$ por cada caracter de S se revisa todo el árbol derivado de colocarlo al principio y luego al final.

\subsection{Enfoque $O(|S|^2)$}
Para esta solución creamos una matriz dp de tamaño $SxS$ donde
donde dp[i][j] representa la cantidad de soluciones que aportaría el caracter i
tomando un total de j caracteres.

Inicializamos esa matriz de la siguiente manera,
iterando por cada uno de los caracteres de S,
Si $T[i] == S[0]$ o si $i > len(T)$ tomar el caracter i con un 
total de i caracteres aportaría una solución al total, esto es dp[i][i] = 1

Luego, iterando por cada uno de los caracteres de S,
se fijarán progresivamente una cantidad k (de 0 a |S|) de caracteres
para cada uno de los restantes caracteres:

si la cantidad de caracteres fijados es mayor que len(S) ya deberá estar la cadena contenida, por tanto se 
considera que fijar el caracter k aportaría todas las soluciones que lo contengan más las soluciones que fijan un caracter más en la solución
de igual forma, si k es menor que len(T) pero T[k] == S[i] se añaden las soluciones que resultan de fijar k+1 caracteres.

Si se toma una cantidad de caracteres con un tamaño mayor o igual que len(T) o S[i] == T[j] se tendrán además la cantidad de soluciones que tome j-1 caracteres de j con k caracteres fijados

Haciendo el proceso anterior se condensa en la posición dp[0][len(S)-1] la cantidad de cadenas diferentes que se pueden formar con T como prefijo con el procedimiento descrito.

Esta solucion itera por los caracteres de S; y por cada uno de estos recorre nuevamente los caracteres de S desde la posición en donde se encuentra hasta el final, esto tendrá una complejidad
de $O(|S| * \sum_1^{|S|})$ que es acotado superiormente por $O(|S|^2)$


\section{Enfoque $O(|T|*|S|)$}
Para pasar a explicar este enfoque es necesario realizar primero unas observaciones.

Existen solo dos formas de generar la cadena:
\begin{enumerate}
    \item Que aparezca T en S de forma consecutiva
    \item Formar T al principio de la nueva cadena
\end{enumerate}

El segundo caso solo es posible si se colocan los caracteres que forman T al principio de la cadena y los que no, al final.

Con esto claro, vamos a pasar a procesar la cadena, lo rimero que nos va a interesar es tomar S al revés

Por cada prefijo de T se recorre S y se guarda cuantas formas hay de generar el prefijo $X_i$ de T para cada posición de S.
Esto es:
Si se tienen las cadenas $S = "abcddacbbccd"$ y $T = "abcd"$
Para $X_1$ ('a')  se guarda en un array la cantidad de formas de generar ese prefijo para cada posición de S;

Esto quedaría: $dp[1] = [1,1,1,1,1,2,2,2,2,2,2,2]$

Con esto se puede entonces generar $dp[2]$ buscando entonces la cantidad de formas de construir $"ab"$
Como ya se tiene la cantidad de cadenas que se pueden generar para el prefijo de tamaño anterior de la siguiente forma:

Para cada posición de S, si el caracter no es el que estoy buscando (el nuevo caracter del prefijo)
se suma la cuenta actual con 0, dado que no se formaron nuevas cadenas en esa posición

En caso de que si sea el caracter deseado, se suma la cuenta actual más la cantidad de formas de generar el prefijo anterior en esta posición.

\begin{table}
    S = "abcddacbbccd"  T = "abcd" \\
    'a' $ \rightarrow$ 1 1 1 1 1 2 2 2 2 2 2 2\\ 
    'ab' $ \rightarrow$ 0 1 1 1 1 1 1 3 5 5 5 5 \\
    'abc' $ \rightarrow$ 0 0 1 1 1 1 2 2 2 7 12 12\\ 
    'abcd' $\rightarrow$ 0 0 0 1 2 2 2 2 2 2 2 14

\end{table}

En este caso por cada caracter de T se recorre toda la cadena S, dejando una complejidad de $O(|T||S|)$

\section{running}

Para correr el archivo, se puede ejecutar:

\texttt{python n\_squared.py}

para utilizarlo como un módulo:

\begin{lstlisting}
    from n_squared import solver
    # se definen las variables S y T
    sol = solver(S,T)
\end{lstlisting}

\section{Testing}
Para el testing de esta implementación se utilizó la biblioteca pytest (necesario instalarla para correr los tests)

Desde la raíz del proyecto correr:
\texttt{python -m pytest}

\end{document}
